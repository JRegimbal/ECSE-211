\documentclass[10pt]{article}

\usepackage{amsmath, amssymb, amsfonts}
\usepackage{gensymb}
\usepackage{geometry}
\usepackage[table]{xcolor}

\geometry{margin=2cm}

\title{Lab 5 - Searching For Objects}
\author{Harley Wiltzer (260690006)\\Juliette Regimbal (260657238)}
\date{October 28, 2016}

%Color definitions
\definecolor{dblue}{rgb}{0.4,0.4,0.6}
\definecolor{lblue}{rgb}{0.8,0.8,1.0}
\definecolor{lyellow}{rgb}{1.0,1.0,0.6}
\definecolor{lgreen}{rgb}{0.7,1.0,0.7}
\definecolor{lred}{rgb}{1.0,0.7,0.7}

\begin{document}
\maketitle
\section{Data/Analysis}
\subsection{Perform at least 10 trials of object recognition using an object other than the
Styrofoam block and note the number of false positives.}
After exposing the robot to 10 different non-Styrofoam objects, only one false positive was
observed. Thus, the success rate for identifying when an object is not the appropriate block was
observed to be 90\%.
\subsection{Repeat the above step, but using the Styrofoam block each time, noting the total number
of false negatives.}
After exposing the robot to the Styrofoam block ten times, the robot successfully identified that it
was a Styrofoam block each time. Thus, no false negatives were observed and the apparent rate it
which the robot can detect a Styrofoam block was 100\%.
\subsection{Run through the search program at least 5 times, recording the average time taken to
	localize, to find a block, and then to travel to the destination. Also estimate the localization
and final destination errors for each trial.}
\subsubsection{Localization}
\begin{figure}[h!t]
\begin{center}
\caption{Localization Statistics}
\rowcolors{2}{lblue}{white}
\begin{tabular}{| c | c | c | c |}
\multicolumn{3}{c}{} \\ \hline
\rowcolor{dblue}
Trial No. & Localization Time (s) & Euclidean Distance Error (cm) & Heading Error (\degree) \\ \hline
1 & 25.9 & 1.7 & 8.3 \\ \hline
2 & 23.5 & 2.0 & 7.4 \\ \hline
3 & 27.0 & 0.4 & 7.6 \\ \hline
4 & 26.2 & 1.3 & 9.7 \\ \hline
5 & 25.5 & 1.3 & 2.0 \\ \hline
\rowcolor{lred}
Average & 24.7 & 1.34 & 7.0 \\ \hline
\end{tabular}
\end{center}
\end{figure}
After 5 trials, the average time taken for the robot to localize was 24.7 seconds. This falls within
the acceptable range for the competition. However, significant error was observed. The average error
in displacement from localization (measured as euclidean distance) was approximately 1.34cm, and the
average error in heading was observed to be approximately 7.0\degree.
\newpage
\subsubsection{Finding a block}
\begin{figure}[h!t]
\begin{center}
\caption{Search Statistics}
\rowcolors{2}{lblue}{white}
\begin{tabular}{| c | c |}
\multicolumn{2}{c}{} \\ \hline
\rowcolor{dblue}
Trial No. & Time (s) \\ \hline
1 & 11.0 \\ \hline
2 & 12.2 \\ \hline
3 & 12.2 \\ \hline
4 & 9.8 \\ \hline
5 & 12.6 \\ \hline
\rowcolor{lred}
Average & 11.56 \\ \hline
\end{tabular}
\end{center}
\end{figure}
Throughout the five trials, the robot took on average 11.56 seconds to find a block (and identify
that it's a Styrofoam block) that was situated approximately in the center of the board, oriented
orthogonally to the coordinate axes. Once again, the robot successfully identified that the block
was a Styrofoam block 100\% of the time.
\subsubsection{Arriving at the final destination}
\begin{figure}[h!t]
\begin{center}
\caption{Destination Statistics}
\rowcolors{2}{lblue}{white}
\begin{tabular}{| c | c | c |}
\multicolumn{3}{c}{} \\ \hline
\rowcolor{dblue}
Trial No. & Time to Destination (s) & Destination Error (cm) \\ \hline
1 & 13.5 & 12.3 \\ \hline
2 & 13.3 & 11.0 \\ \hline
3 & 13.4 & 10.9 \\ \hline
4 & 13.4 & 14.4 \\ \hline
5 & 13.0 & 6.3 \\ \hline
\rowcolor{lred}
Average & 13.32 & 11.0 \\ \hline
\end{tabular}
\end{center}
\end{figure}
After finding the Styrofoam block (positioned and oriented as described above), the robot took, on
average, 13.32 seconds to position the block in the target destination. Unfortunately, there was
significant error in the block's final position. This error ranged from 6.3cm to 14.4cm, averaging
at approximately 11.0cm.

\section{Observations and Conclusions}
\subsection{What differences, if any, were observed in the behavior/performance of your earlier code
when combined in a larger system? Explain any discrepancies}.
Odometry and localization suffered painfully upon their integration to the larger system. This was
mainly due to the lack of a color sensor for those purposes. Since only one color sensor is
available and needed to be used for analyzing objects, it could not be used to detect gridlines
without sacrificing a lot of time to implement its own motor and control. As a consequence, the
odometry had no sensory feedback, and this led the error in odometry to accrue to a large extent.
This was observed when the robot misplaced the Styrofoam block by an average of 11.0cm. Furthermore,
the localization routine had to resort to using only the ultrasonic sensor, and couldn't use the
color sensor to further improve its measurements. The ultrasonic sensor is susceptible to much more
noise than the color sensor, so its readings are not as accurate. To diminish this effect, a median
filter was implemented for the ultrasonic readings, however the ultrasonic sensor still could not
measure distances as accurately as the light sensor did in Lab 4. Thus, larger errors in heading and
position after localization were observed.
\subsection{How reliable was your object detection? What factors influence the reliability of object
	detection? Where would you expect your code to break down? What steps can you take to make
detection more robust?}
In the conditions under which the robot was tested, the object detection was extremely reliable. As
seen in \textbf{Data Analysis}, no false negatives occurred, and false positives were observed very
infrequently. The object detection implemented for this robot was based on the ratio of red to green
color values, and was far from robust. The code would likely break down in conditions where ambient
light is different from that in the lab. To make the detection more robust, ambient light can be
taken into account to offer some prediction as to how the expected of the color of the Styrofoam
block the be. Then, colors may be interpreted as three-dimensional vectors, and the classification
of objects can be done by analyzing the euclidean distance between two colors. Furthermore, instead
on relying solely on color, the object detection can also attempt to measure the dimensions of the
block that it is detecting. Then, these dimensions may be compared with those of the Styrofoam
block.
\subsection{What aspect of this lab did you find most difficult? What aspect of this lab did you
find most surprising or unexpected?}
The most difficult parts of this lab were reliably searching for objects and reducing the likelihood
of the robot crashing into obstacles. The searching algorithm was difficult to conceive, because due
to the lack of a rotating ultrasonic sensor, the search algorithm had to ensure that the robot
approaches the obstacles it detects at an appropriate angle. This was very difficult to code and
test because so many cases are possible. Furthermore, as the robot dragged the block from behind,
its ``effective radius'' - that is to say, the furthest point of the robot from its center of
rotation, was very large. Therefore, when the robot turned, it covered an enormous amount of area.
It was very difficult to reduce the occurrences of the robot hitting obstacles when it was turning
for this reason. Surprisingly, the robot was exceptionally good at identifying obstacles. As
mentioned previously, errors in obstacle identification were rare, which was unexpected given
previous bad experiences with other sensors during this lab and previous experiments.
\end{document}
