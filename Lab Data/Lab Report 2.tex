% !TEX TS-program = pdflatex
% !TEX encoding = UTF-8 Unicode

% This is a simple template for a LaTeX document using the "article" class.
% See "book", "report", "letter" for other types of document.

\documentclass[11pt]{article} % use larger type; default would be 10pt

\usepackage[utf8]{inputenc} % set input encoding (not needed with XeLaTeX)

%%% Examples of Article customizations
% These packages are optional, depending whether you want the features they provide.
% See the LaTeX Companion or other references for full information.

%%% PAGE DIMENSIONS
\usepackage{geometry} % to change the page dimensions
\geometry{a4paper} % or letterpaper (US) or a5paper or....
% \geometry{margin=2in} % for example, change the margins to 2 inches all round
% \geometry{landscape} % set up the page for landscape
%   read geometry.pdf for detailed page layout information

\usepackage{graphicx} % support the \includegraphics command and options

% \usepackage[parfill]{parskip} % Activate to begin paragraphs with an empty line rather than an indent

%%% PACKAGES
\usepackage{booktabs} % for much better looking tables
\usepackage{array} % for better arrays (eg matrices) in maths
\usepackage{paralist} % very flexible & customisable lists (eg. enumerate/itemize, etc.)
\usepackage{verbatim} % adds environment for commenting out blocks of text & for better verbatim
\usepackage{subfig} % make it possible to include more than one captioned figure/table in a single float
% These packages are all incorporated in the memoir class to one degree or another...

%%% HEADERS & FOOTERS
\usepackage{fancyhdr} % This should be set AFTER setting up the page geometry
\pagestyle{fancy} % options: empty , plain , fancy
\renewcommand{\headrulewidth}{0pt} % customise the layout...
\lhead{}\chead{}\rhead{}
\lfoot{}\cfoot{\thepage}\rfoot{}

%%% SECTION TITLE APPEARANCE
\usepackage{sectsty}
\allsectionsfont{\sffamily\mdseries\upshape} % (See the fntguide.pdf for font help)
% (This matches ConTeXt defaults)

%%% ToC (table of contents) APPEARANCE
\usepackage[nottoc,notlof,notlot]{tocbibind} % Put the bibliography in the ToC
\usepackage[titles,subfigure]{tocloft} % Alter the style of the Table of Contents
\renewcommand{\cftsecfont}{\rmfamily\mdseries\upshape}
\renewcommand{\cftsecpagefont}{\rmfamily\mdseries\upshape} % No bold!

%%% END Article customizations

%%% The "real" document content comes below...

\title{Design Principles and Methods - Odometry Lab Report}
\author{Harley Wiltzer (260690006)\\Juliette Regimbal (260657238)}
\date{October 6, 2016} % Activate to display a given date or no date (if empty),
         % otherwise the current date is printed 
\pagenumbering{gobble}
\begin{document}
\maketitle

\section{Objective}
To determine the accuracy of the implemented odometry system, and implement a simple correction using a light sensor.
\section{Method}
\begin{enumerate}
\item In the file Odometer.java, implement code that performs the task of an odometer as
described in the Odometry tutorial and in class. You should only need to add member variables
to and modify the run() method of the Odometer class.
\item Run the robot in a 3-by-3 tile square (where one tile is 30.48 cm in linear dimension) using the
provided code and tweak the leftRadius, rightRadius, and width parameters passed
to the SquareDriver.drive() method in Lab2.java until the robot returns
(approximately) to its starting position. If your left and right wheel motors are not connected to
motor ports A and B respectively, you may need to also modify those parameters. The call to
SquareDriver.drive() is shown below.
\end{enumerate}
\section{Data}
\begin{center}
\begin{tabular}{ | c | c | c | }
\multicolumn{3}{c}{Offset from Origin - Correction Disabled} \\ \hline
X (cm) & Y (cm) & $\theta$ (rad)\\ \hline
-0.42 & -0.66 & 0.01 \\ \hline
-0.67 & -0.86 & 0.01 \\ \hline
-0.84 & -0.84 & 0.01 \\ \hline
-0.45 & -0.46 & 0.01 \\ \hline
-0.64 & -0.82 & 0.01 \\ \hline
-0.82 & -0.45 & 0.01 \\ \hline
-0.69 & -0.66 & 0.01 \\ \hline
-0.39 & -0.44 & 0.01 \\ \hline
-0.84 & -0.66 & 0.01 \\ \hline
-0.64 & -0.63 & 0.01 \\ \hline
\end{tabular}
\end{center}

\begin{center}
\begin{tabular}{ | c | c | c | }
\multicolumn{3}{c}{Offset from Origin - Correction Enabled} \\ \hline
X (cm) & Y (cm) & $\theta$ (rad)\\ \hline
-0.04 & -0.09 & 0.01 \\ \hline
-0.28 & -0.18 & 0.01 \\ \hline
-0.18 & -0.19 & 0.01 \\ \hline
-0.28 & -0.27 & 0.01 \\ \hline
-0.31 & -0.18 & 0.01 \\ \hline
-0.19 & -0.20 & 0.01 \\ \hline
-0.27 & -0.13 & 0.01 \\ \hline
-0.18 & -0.25 & 0.01 \\ \hline
-0.08 & -0.12 & 0.01 \\ \hline
-0.17 & -0.11 & 0.01 \\ \hline
\end{tabular}
\end{center}

\section{Data Analysis}

\section{Observations and Conclusion}

\section{Further Improvements}

\end{document}
