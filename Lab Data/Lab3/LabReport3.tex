\documentclass[11pt]{article}

\usepackage{graphicx}
\usepackage{amsmath}
\usepackage{amssymb}
\usepackage{amsfonts}
\usepackage{gensymb}
\usepackage{sectsty}
\usepackage{geometry}
\usepackage[ampersand]{easylist}

\geometry{margin=1cm}

\title{Design Principles and Methods - Navigation Lab Report}
\author{Harley Wiltzer (260690006)\\Juliette Regimbal (260657238)}
\date{October 13, 2016}
\pagenumbering{gobble}
\begin{document}
\maketitle
\pagenumbering{arabic}

\section{Objective}
To design a software system that allows a (main) thread of execution to command it to drive the robot to an absolute position on the field while avoiding obstacles, by use of the odometer and an ultrasonic sensor.
\section{Method}
\begin{easylist}[itemize]
& Using the nagivation tutorial provided, create a class that extends \texttt{Thread} (or, if you want to be very resourceful, a class that implements \texttt{TimeListener}) to control the robot's motors to drive to a specified point on the field, given in Cartesian coordinates. Your class should at least implement the following public methods:
&& \texttt{void travelTo(double x, double y)}\\This method causes the robot to travel to the absolute field location (x, y). This method should continuously call \texttt{turnTo(double theta)} and then set the motor speed to forward(straight). This will make sure that your heading is updated until you reach your exact goal. (This method will poll the odometer for information.)
&& \texttt{void turnTo(double theta)}\\This method causes the robot to turn (on point) to the absolute heading \texttt{theta}. This method should turn a MINIMAL angle to its target.
&& \texttt{boolean isNavigating()}\\This method returns true if another thread has called \texttt{travelTo()} or \texttt{turnTo()} and the method has yet to return; false otherwise.
& Adjust the parameters of your controller, if any, to minimize the distance between the desired
destination and the final position of the robot, while simultaneously minimizing the number of
oscillations made by the robot around its destination before stopping.
& Write a program to travel to the waypoints (60, 30), (30, 30), (30, 60), and (60, 0) in that
order. Test this program ten (10) times, and record the position the robot a) as reported by the
odometer, and b) as measured on the field.
& Modify your code to use the ultrasonic sensor to detect an obstacle (a cinder block) and avoid
collision with it. You may mount the ultrasonic sensor as you wish, and may use your wall follower
code if desired. Additionally, you may create another class to perform this function.
& Write a program to, with obstacle avoidance, travel to the waypoints (0, 60) and (60, 0) in that
order. To demonstrate the effectiveness of your obstacle avoidance, the TA will first trun your
robot's program to travel that path without obstacles, then place a single cinder block somewhere
along the path (though not at a waypoint), and run your robot's program again.
\end{easylist}
\newpage
\section{Data}
\begin{figure}[h!t]
\begin{center}
\caption{Final Position of Robot (cm) - Part 1}
\begin{tabular}{| c | c | c | c | c | c | c |}
\multicolumn{4}{c}{} \\ \hline
Trial No. & x-odometer (cm) & y-odometer (cm) & x-actual (cm) & y-actual (cm) & x-error (cm)  &
y-error (cm) \\ \hline
1 & 60.92 & -1.80 & 61.8 & 0.4 & -0.88 & -2.20 \\ \hline
2 & 60.84 & -1.52 & 63.0 & -0.2 & -2.16 & -1.32 \\ \hline
3 & 60.61 & -0.87 & 61.9 & 0.5 & -1.29 & -1.31 \\ \hline
4 & 60.88 & -1.51 & 62.2 & -0.3 & -1.32 & -1.21 \\ \hline
5 & 60.94 & -1.66 & 61.5 & -0.7 & -0.56 & -0.96 \\ \hline
6 & 60.86 & -1.58 & 62.4 & -0.4 & -1.54 & -1.18 \\ \hline
7 & 59.92 & 0.20 & 61.0 & 1.6 & -1.08 & -1.40 \\ \hline
8 & 60.60 & -0.65 & 61.2 & 0.3 & -0.60 & -0.95 \\ \hline
9 & 61.04 & -1.86 & 61.5 & -1.2 & -0.46 & -0.66 \\ \hline
10 & 61.08 & -2.04 & 63.0 & -0.2 & -1.92 & -1.84 \\ \hline
Mean & 60.77 & -1.33 & 61.95 & -0.02 & -1.18 & -1.30 \\ \hline
St. Dev. & 0.34 & 0.69 & 0.70 & 0.77 & 0.58 & 0.44 \\ \hline
\end{tabular}
\end{center}
\end{figure}
\section{Data Analysis}
\subsection{Calculations of Mean and Standard Deviation for Error in Odometer Readings}
\begin{equation}
	\mbox{Mean} = \frac{1}{n} \sum_{k=1}^{n}\mathcal{E}_k
\end{equation}
\begin{equation}
	\mbox{Standard Deviation} = \sqrt{\frac{\sum_{k=1}^{n}(\mathcal{E}_k-\mu_{\mathcal{E}})^2}{n-1}}
\end{equation}
In formulae (1) and (2), $n$ represents the number of trials for which data was acquired,
$\mu_{\mathcal{E}}$ represents the mean of the error, and
$\mathcal{E}_k$ represents the error in measurement in some coordinate axis. Below, the calculations
for these values are explicitly shown.
\subsubsection{Calculations of the means}
%\textbf{Mean of Error in Measurement in $x$ Position}
\begin{equation*}
	\begin{aligned}
		\overline{x} &= \dfrac{-0.88-2.16-1.29-1.32-0.56-1.54-1.08-0.60-0.46-1.92}{10}
		&= -1.18 \mbox{cm}
	\end{aligned}
\end{equation*}
\begin{equation*}
	\begin{aligned}
		\overline{y} &= \dfrac{-2.20-1.32-1.31-1.21-0.96-1.18-1.40-0.95-0.66-1.84}{10}
		&= -1.30 \mbox{cm}
	\end{aligned}
\end{equation*}
\subsubsection{Calculations of the standard deviations}
\resizebox{\linewidth}{!}{
	\begin{minipage}{\linewidth}
		\begin{align*}
		s_x &=
		\sqrt{\dfrac{(-0.88-(-1.18))^2+(-2.16-(-1.18))^2+(-1.29-(-1.18))^2+(-1.32-(-1.18))^2+(-0.56-(-1.18))^2+(-1.54-(-1.18))^2+(-1.08-(-1.18))^2+(-0.60-(-1.18))^2+(-0.46-(-1.18))^2+(-1.92-(-1.18))^2}{9}}
		\\
		&= 0.58\mbox{cm}
		\end{align*}
	\end{minipage}
}\\
\resizebox{\linewidth}{!}{
	\begin{minipage}{\linewidth}
		\begin{align*}
		s_y &=
		\sqrt{\dfrac{(-2.20-(-1.30))^2+(-1.32-(-1.30))^2+(-1.31-(-1.30))^2+(-1.21-(-1.30))^2+(-0.96-(-1.30))^2+(-1.18-(-1.30))^2+(-1.40-(-1.30))^2+(-0.95-(-1.30))^2+(-0.66-(-1.30))^2+(-1.84-(-1.30))^2}{9}}
		\\
		&= 1.44\mbox{cm}
		\end{align*}
	\end{minipage}
}
\subsection{Are the errors present as a result of the odometer or the navigator?}
The errors are caused by both the odometer and the navigator. As seen in the Odometer experiment, there
is significant slippage occuring when the wheels spin. As a consequence of such slippage, there must
be error in the measurement of the $x$ and $y$ position of the robot, as well as in the measurement
of its heading. \\
However, due to the design of the navigator, more error emerges. Since the navigator
must correct its angle continously, one cannot simply tell the navigator to travel some distance in
a given direction. Instead, the navigator is programmed to move on to the next waypoint when it is
within a certain range of the desired target position. Therefore, with this design, error is
inherent and even \textit{allowed}, as a tradeoff for better heading correction and a more dynamic
controller. The result is that
while the robot can very accurately reach the waypoint in the appropriate direction, it only gets
to within a certain error margin of the waypoint. These errors, however, are independent from that of
the odometer error. Although the robot may not reach the exact waypoint, the odometer \textit{knows}
the robot has not reached the exact waypoint - that is to say, although there is error in the robots
position due to the design of the navigator, this does not hinder the performance of the odometer.
\\
Furthermore, since the robot is continuously correcting its heading, the wheels undergo a lot of
acceleration (in order to change the robot's direction). Although usually the corrections are very
small, in some situations the corrections may be large enough to cause significant slipping (because
of the acceleration) which can lead to significant error in the odometer readings.\\
Finally, another source of error lies in the model of the robot. Odometer calculations are done
given certain measurements, such as the radii of the wheels of the robot and the distance between
them. The controller of the robot assumes that these are perfectly accurate, which is not true in
reality, thus any error on those measurements leads to error in the odometer's performance. Also, as
the wheels accelerate, they undergo stress which can lead to deformation. The model of the robot in
the controller assumes perfectly circular wheels, and such an assumption grows less and less
accurate as the wheels wear out, thus introducing more error in the odometer.

\section{Observations and Conclusion}
\subsection{Explain the operation of the controllers for navigation. How accurately does it move the
robot to its destination? How quickly does it settle on its destination?}
The controller for navigation works by continuously correcting the heading of the robot and having
it move forward until it is within a certain euclidean distance from the waypoint. The heading
correction works by using the waypoint coordinates and the odometer's
readings to calculate by which angle the robot must rotate to be facing its destination, using
Java's \texttt{atan2} method. To avoid
constant, tiny corrections (which would result in very jerky behavior and possibly more slippage),
the robot only corrects its heading when it is outside some threshold. \\
The controller for obstacle avoidance is actually implemented in the same class as that for basic
navigation. The difference is that when the ultrasonic sensor detects that it is close to a wall,
the normal operation of the path follower is skipped, and the robot turns its ultrasonic sensor some
$\theta$ degrees, and rotates on point $-\theta$ degrees to keep the sensor in the same relative
position. At this point, the robot's heading $\Theta$ is stored, and it contours the wall using a
Bang-Bang controller until its heading is within a small margin of $\Theta + 180$ degrees. Next, the
robot turns its sensor back $-\theta$ degrees, and turns $\theta$ degrees while moving one of its
wheels faster than the other so as to move away from the wall as it rotates, at which point it can
return to its normal navigation method.\\
After tweaking the threshold value associated with the minimum error for which the robot corrects
its heading in \texttt{turnTo()}, the robot was found to settle relatively quickly on its target. In
the worst case, the robot was seen to make up to three oscillations near a waypoint, however its
accuracy and precision were sublime. Furthermore, the threshold corresponding to the maximum
euclidean distance from a waypoint for which the robot moves on to the next waypoint was tweaked
until the robot reached its destination within an appropriate error margin. As seen in section
\textbf{3}, the robot was always accurate to within 3cm from its target destination.

%This helps the robot settle
%quickly on its destination without sacrificing much in terms of accuracy. The robot was observed to
%navigate to comfortably within 3 centimeters of its destination in every trial.

\subsection{How would increasing the speed of the robot affect the accuracy of the robot? What is
the main source of error in navigation (and odometry)?}
The main source of error in odometry and odometry-based navigation is wheel slippage. Since the
position of the robot is tracked by using wheel rotation with an accurate physical model of the
chassis, any deviation from its expected behavior - such as wheel slipping - will introduce error
into odometry and thus navigation. Since the robot makes corrections in navigation and changes paths
to go to different waypoints, a higher traveling speed would require more acceleration. With the
wheels accelerating more, the likelihood that they will slip more often and for longer when making
adjustments to wheel direction and speed increases. And as this error increases, the overall
accuracy in navigation decreases.

\section{Further Improvements}
To decrease error, higher friction wheels could be substituted to make it more difficult for the
robot to slip during normal use. 
Predictions could be made by finding if there is a constant amount of error introduced over a
certain distance (for example, the robot could be consistently +0.5cm off per 100cm travelled). This
correction could be factored into odometry in software, allowing for more accurate positioning in
navigation. Furthermore, reducing the speed of the robot would in turn reduce the amount of
acceleration of the robot, effectively reducing the amount of slipping of the wheels. Finally a
method of landmark based correction could be implemented, similar to how the light
sensor detected evenly-spaced lines in Lab 2. Landmark detection would provide an absolute distance
from either a point or line on the track, allowing for the robot to reset its odometry readings and
essentially zero the error accumulated through travel.
\end{document}
