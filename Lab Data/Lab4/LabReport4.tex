\documentclass[11pt]{article}
\usepackage{graphicx}
\usepackage{amsmath}
\usepackage{amssymb}
\usepackage{amsfonts}
\usepackage{gensymb}
\usepackage{sectsty}
\usepackage{geometry}
\usepackage[ampersand]{easylist}
\usepackage[table]{xcolor}

\geometry{margin=1cm}

\title{Design Principles and Methods - Localization Lab Report}
\author{Harley Wiltzer (260690006)\\Juliette Regimbal (260657238)}
\date{October 20, 2016}
\pagenumbering{gobble}

\begin{document}
\maketitle
\pagenumbering{arabic}

%Color definitions
\definecolor{dblue}{rgb}{0.4,0.4,0.6}
\definecolor{lblue}{rgb}{0.8,0.8,1.0}
\definecolor{lyellow}{rgb}{1.0,1.0,0.6}
\definecolor{lgreen}{rgb}{0.7,1.0,0.7}
\definecolor{lred}{rgb}{1.0,0.7,0.7}

\section{Objective}
To use the ultrasonic sensor and light sensor to accurately navigate the robot to a known (initial) position and orientation on the field.
\section{Method}
\begin{enumerate}
\item Put you odometer and navigation classes created in Labs 2 and 3 in the Odometer.java and Navigation.java files. Working odometer and navigation classes are also provided in case you had failed to complete either lab.
\item Fill in the code for the class called \texttt{USLocalizer}. This class actually contains two different localization routines, each of which can be implemented and tested separately. You may need to create extra functions in your Navigation in order to move as you require.
\item Test each localization routine ten times using random starting orientations (but the same starting position, notably in the corner square) and record the error in the final orientation of the robot. Compute the mean and standard deviation for each routine.
\item Based on the standard deviations from (2), determine the best ultrasonic sensor-based localization method for your robot. Use this one for the rest of the lab, but do not remove the code for the other, as you will need to submit it. Also, correct the appropriate constant in your code to make the mean error 0. You should not need to do any additional tests to confirm that your correction in fact made it 0.
\item Fill in the code for the class called \texttt{LightLocalizer}. You need not test the accuracy of this part of the localization. You'll require some trigonometric equations as outlined in the Localization tutorial. Follow the process that the tutorial demonstrates. Also when you've found (0, 0) and 0$\degree$, travel to (0, 0) and turn to 0$\degree$.
\item Demonstrate to a TA the correct operation of your robot's localization. The TA will choose the starting orientation of your robot. As can be inferred from the comments in the provided code, your robot should: a) use the ultrasonic sensor and the routine you developed and tested in (1), (2), and (3) to find and rotate to an approximatation of 0$\degree$, b) drive to the point specified in the Localization tutorial. Begin rotating and clocking angles. c) Compute the trigonometric values for the robot's heading and the (0, 0) point, and d) Travel to (0, 0, 0).
\end{enumerate}
\newpage
\section{Data}
\begin{figure}[h!t]
\begin{center}
\caption{Offset from 0$\degree$ - Falling and Rising Edge}
\rowcolors{2}{lblue}{white}
\begin{tabular}{| c | c | c |}
\multicolumn{3}{c}{} \\ \hline
\rowcolor{dblue}
Trial No. & Falling Edge Error ($\degree$) & Rising Edge Error ($\degree$) \\ \hline
1 & 9.6 & 3.8 \\ \hline
2 & 6.5 & 3.0 \\ \hline
3 & 6.1 & -1.2 \\ \hline
4 & 14.7 & -1.5 \\ \hline
5 & 4.6 & -1.9 \\ \hline
6 & 5.7 & 0.4 \\ \hline
7 & 5.0 & -1.9 \\ \hline
8 & 3.8 & -16.7 \\ \hline
9 & 2.7 & 1.5 \\ \hline
10 & 15.5 & -1.2 \\ \hline
\rowcolor{lyellow}
Mean & 7.4 & -1.6 \\\hline
\rowcolor{lred}
Standard Deviation & 4.4 & 5.7 \\ \hline
\end{tabular}
\end{center}
\end{figure}
\section{Error Calculations}
\begin{equation}
	\mbox{Mean} = \frac{1}{n} \sum_{k=1}^{n}\mathcal{E}_k
\end{equation}
\begin{equation}
	\mbox{Standard Deviation} = \sqrt{\frac{\sum_{k=1}^{n}(\mathcal{E}_k-\mu_{\mathcal{E}})^2}{n-1}}
\end{equation}
\subsection{Falling Edge}
Using formula (1), the mean error of the falling edge localization routine will be calculated.
\begin{equation*}
	\begin{aligned}
		\mu_F &= \dfrac{9.6+6.5+6.1+14.7+4.6+5.7+5.0+3.8+2.7+15.5}{10} \\
		&= 7.4 \degree
	\end{aligned}
\end{equation*}
Next, using formula (2), the standard deviation of these measurements will be calcuated.\\
\resizebox{\linewidth}{!}{
	\begin{minipage}{\linewidth}
		\begin{align*}
			\sigma_F &=
			\sqrt{\dfrac{(9.6-7.4)^2+(6.5-7.4)^2+(6.1-7.4)^2+(14.7-7.4)^2+(4.6-7.4)^2+(5.7-7.4)^2+(5.0-7.4)^2+(3.8-7.4)^2+(3.8-7.4)^2+(2.7-7.4)^2+(15.5-7.4)^2}{9}}
			\\
			&= 4.4 \degree
		\end{align*}
	\end{minipage}
}
\subsection{Rising Edge}
Using formula (1), the mean error of the falling edge localization routine will be calculated.
\begin{equation*}
	\begin{aligned}
		\mu_R &= \dfrac{3.8+3.0-1.2-1.5-1.9+0.4-1.9-16.7+1.5-1.2}{10} \\
		&= -1.6 \degree
	\end{aligned}
\end{equation*}
Next, using formula (2), the standard deviation of these measurements will be calcuated.\\
\resizebox{\linewidth}{!}{
	\begin{minipage}{\linewidth}
		\begin{align*}
			\sigma_R &=
			\sqrt{\dfrac{(3.8+1.6)^2+(3.0+1.6)^2+(-1.2+1.6)^2+(-1.5+1.6)^2+(-1.9+1.6)^2+(0.4+1.6)^2+(-1.9+1.6)^2+(-16.7+1.6)^2+(1.5+1.6)^2+(-1.2+1.6)^2}{9}}
			\\
			&= 4.4 \degree
		\end{align*}
	\end{minipage}
}
\section{Observations and Conclusion}
\subsection{Which of the two localization routines performed the best? Which performed the worst?
What factors contributed to the performance (or lack thereof) of each method?}
\par The falling edge localization routine, before any adjustments were made, had a higher mean error
than the rising edge localization, meaning the falling edge routine was less accurate. However, the
standard deviation of the falling edge routine was lower than that of the rising edge routine,
meaning the falling edge localization was more precise. Because of this low standard deviation, the
falling edge localization was further corrected to rotate by the mean error once the initial
localization was complete, so as to reduce the mean to approximately 0 degrees. This was reliable
since the falling edge routine was more consistent, so the error in each trial was closer to the
mean, and it can be expected then that this extra rotation will consistently localize the robot with
minimal error. 

\par There was a plethora of factors that hindered the performance of both localization methods. Firstly,
it was extremely important for the robot to start with its center of rotation on the 45\degree 
 diagonal. Deviance from this starting point resulted in much larger error than when the robot was
more properly placed, as errors up to 22.1\degree  were observed in such circumstances. Furthermore,
noise from the ultrasonic sensor was particularly detrimental to the performance of both falling
edge and rising edge localization, as the ultrasonic sensor was detecting walls that did not exist.
This turned out to be more of an issue during the falling edge localization. However, this error was
reduced by filtering the ultrasonic sensor data, and by only checking for walls after a certain
heading difference from the last wall. Finally, rising edge localization experienced some errors
presumably due to the ultrasonic sensor interfering with itself. The rising edge localization method
has the robot facing the wall most of the time, so the pings of the ultrasonic sensor may have been
interfering with the reflections of other pings off the wall.

\subsection{Why does the light sensor provide a more accurate means of localization than the
ultrasonic sensor?}
\par The localization using the light sensor detects 4 angles - one for each line - before
determining the heading correction. The ultrasonic sensor only detects 2 angles, one for each wall.
Along with this the light sensor is less susceptible to noise than the ultrasonic sensor. While the
ultrasonic sensor often reports wildly disparate values and needs at least basic filtering to be at
all useful. Even then the ultrasonic sensor is susceptible to noise in its environment, causing
false positives and introducing error into the correction. The light sensor, however, has much more
stable behaviour. It is a fixed distance away from the odometer's center and the light sensor is far
less prone to false positives and does not even require a filter. For these reasons localization
using the light sensor produces a more reliably accurate correction than localization using the
ultrasonic sensor. Furthermore, the conditions for successful localization with the light sensor are
far less constrained than that with the ultrasonic sensor. For the light sensor localization to
work, the robot must be positioned some distance (depending on the design of the robot) from the
origin, in the third quadrant. No matter what its position or orientation are at this point, the
light sensor localization should be successful. With the ultrasonic sensor, the robot's center of
rotation must be very close to the diagonal ranging from the corner of the wall to $(0,0)$. Small
deviations from this position can result in large error in localization. Therefore, the accuracy of
the ultrasonic sensor localization routine is limited much more by human accuracy (placing the robot
on the $45 \degree$ line) than the light sensor routine, which explains why the light sensor
localization is often much more accurate and precise.
\subsection{Propose a means of determining (approximately) the initial position of the robot using
the ultrasonic sensor. Why is detecting minima with the ultrasonic sensor problematic?}
\par The approximate initial position of the robot could be calculated using the minimal distance
between the robot and each wall. Ideally, this will report an accurate distance in centimeters from
each wall, easily allowing for the $x$ and $y$ coordinates to be set (given the dimensions of the grid).
In reality, this method has issues
stemming from the inaccuracies with the ultrasonic sensor described in section \textbf{6.1}. With so
much error, multiple passes and filtering would be necessary to determine the minima, and even then
the coordinates likely be off be a few centimeters. This process would be time consuming and would
not result in accurate localization, and shouldn't be used with the ultrasonic sensor provided.

\section{Further Improvements}
\subsection{Propose a way to avoid small errors more accurately than a clipping filter.}
To avoid small errors more acurately, there are several methods that can be used as opposed to a
clipping filter. Notably, one may implement a \textit{median filter}, which is very effective at
eliminating the effects of noise for a signal from an ultrasonic sensor. It works as follows: chose
some \textit{window} size $\epsilon$. Every time new data is fetched, the median of the last
$\epsilon$
readings is calculated. If the newest data is greater than this median, it is replaced by the
median.

\subsection{Propose a sensor design that would result in a more accurate and reliable reading than
an ultrasonic sensor.}
\par One alternative sensor design that would yield more accurate results would be to equip the
robot with an infrared sensor and, on the field, place infrared LEDs on the $x$ and $y$ axes. The
robot could search for the position where both LEDs can be detected - (0, 0) - and then rotate
itself to a heading of 0$\degree$ since the LEDs would be directed in the $+x$ and $+y$ directions.
Using light rather than sound prevents other sources of noise from interfering with measurements
taken, and the infrared LEDs would produce a large spike when in the correct location. With such a
spike, the measurements could be interpreted as being binary - detecting or not detecting - which
removes or at the very least simplifies the requirements for filtered input. The heading correction
would only be as accurate as the odometer, but a poorly designed odometer would create far greater
problems than inaccurate localization. The main drawback of this method is that the robot would be
dependent on the LEDs for localization, and would not be able to use the field in its current form.
The infrared light produced by the LEDs would also have to be narrowly focused so their intersection
is only at the (0, 0) point. With no unique landmarks on the field, a combination of inputs using
the light sensor to detect grid lines and the ultrasonic sensor to detect the boundary of the field
would have to be used. Such a method of localization would be more complicated and error prone.

\subsection{Propose another form of localization other than rising edge and falling edge.}
\par Instead of detecting rising or falling edges, a method akin to that described in \textbf{5.3}
can be used. First, the robot would store headings associated with minimal distances (detected by
the ultrasonic sensor) to the walls, these would correspond to the heading for which the robot is
parallel to the negative $x$ axis, $\theta_x$, and the heading for which the robot is facing the
negative $y$ direction, $\theta_y$. Furthermore, the values of those minima will be stored as
$\delta_x$ and $\delta_y$ respectively. With this data, the robot knows (approximately) its position
relative to the corner. Given the length of a gridline, $\aleph$, the robot updates its odometer
position to $(\aleph - \delta_x,\aleph - \delta_y)$. For now, the robot's heading is not
altered. Next, the robot calls \texttt{turnTo($\theta_x
+ 180 \degree$)}, which would orient the robot to the \textit{positive} $x$ axis, at which point it
can set its heading to 0. Note, the $\theta_x + 180\degree$ in the \texttt{turnTo()} call is based
on the fact that $\theta_x$ is the heading of the robot when it is parallel to the \textit{negative}
$y$ axis by definition, thus the robot must rotate another $180\degree$ to be facing the positive
$x$ axis.
\par It should be noted, however, that this method of localization has its disadvantages. As
described in \textbf{5.3}, detecting minima of distance with the ultrasonic sensor can be quite
problematic, therefore the data that is crucial to the localization in this method may be convoluted.
%\par Instead of detecting either rising edges or falling edges, the robot can detect one of each. For
%example, the robot could choose one direction to rotate in, detect a large enough change in distance
%(this can represent a rising edge \textit{or} a falling edge), and continue rotating in that
%direction (as opposed to changing directions in the rising and falling edge methods). At this point,
%the robot will eventually detect another large change in distance (corresponding to a falling edge
%or a rising edge) and make trigonometric calculations based on the two \textit{edges} that it
%detects. The benefit of this routine is that the robot never needs to switch direction, so there is
%no acceleration required to make that switch, effectively reducing the amount of wheel slippage.

\end{document}
